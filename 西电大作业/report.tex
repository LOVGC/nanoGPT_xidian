\documentclass[11pt]{article}
\usepackage[UTF8]{ctex}
\usepackage{geometry}
\usepackage{graphicx}
\usepackage{booktabs}
\usepackage{tabularx}
\usepackage{float}
\usepackage{hyperref}

\geometry{a4paper, margin=1in}
\graphicspath{{./}}

\title{NanoGPT-Edge 项目报告}
\author{张三(学号:123)}
\date{\today}

\begin{document}
\maketitle

\section{项目概述}
本项目从零实现一个小于 50M 参数的 Decoder-only Transformer,并完成 MoE、SFT、量化(PTQ)与 RAG 全流程。目标是在端侧可部署的前提下,提升指令跟随能力并减少“幻觉”。

\section{模型架构与设计分析}
\subsection{整体架构}
模型采用 Decoder-only Transformer。核心组件包括:RMSNorm、RoPE、SwiGLU 以及 MoE(Top-2 Router)。

\begin{figure}[H]
\centering
\includegraphics[width=0.95\linewidth]{Architecture.png}
\caption{模型整体架构示意图}
\end{figure}

\subsection{MoE 路由机制}
在 Transformer Block 的 FFN 位置替换为 MoE,Router 采用 Top-2 路由,在提高模型容量的同时保持推理代价可控。

\begin{figure}[H]
\centering
\includegraphics[width=0.85\linewidth]{MoE.png}
\caption{MoE 路由结构示意图}
\end{figure}

\section{Pre-training 训练分析}
\subsection{数据与目标}
预训练数据集使用 \texttt{roneneldan/TinyStories},该数据集为纯英文,适合小模型快速收敛。训练目标为 Next Token Prediction(交叉熵)。

\subsection{训练曲线与收敛性}
\begin{figure}[H]
\centering
\includegraphics[width=0.9\linewidth]{pretrain_loss.png}
\caption{Pretrain Loss 曲线}
\end{figure}

\textbf{分析:} Loss 下降趋势明显,模型已学到基础语法与句法结构。给定 "Once upon a time" 能生成连贯故事,满足预训练验收标准。

\section{SFT 训练分析}
\subsection{数据与目标}
SFT 数据集使用英文指令数据(Alpaca + Dolly + SQuAD + SQuAD-v2),目标是将模型从“故事续写”转为“指令跟随”。

\subsection{训练曲线}
\begin{figure}[H]
\centering
\includegraphics[width=0.9\linewidth]{sft_loss.png}
\caption{SFT Loss 曲线}
\end{figure}

\textbf{分析:} SFT Loss 很快下降并趋近极低,但生成质量并未同步提升,这是典型的 teacher-forcing 与 free-run mismatch。训练阶段模型在“正确前缀”下预测良好,但生成时首 token 偏离会迅速发散。

\subsection{SFT 失败原因分析(重点)}
\begin{itemize}
\item \textbf{训练过多 / Loss 过低导致“学傻”}:训练步数过多,模型过拟合小样本指令,原有语言建模能力退化(catastrophic forgetting)。
\item \textbf{学习率过大 + 训练太久}:SFT 是“微调”,学习率不宜过大,否则容易冲掉预训练知识。
\item \textbf{数据量太小}:小数据情况下不宜训练太猛,否则过拟合与遗忘被放大。
\item \textbf{语言分布偏移}:预训练为纯英文,fine-tune 混入中文会显著恶化表现。
\item \textbf{验证建议}:可在 SFT 后回测预训练集 PPL,量化遗忘程度。
\end{itemize}

\section{MoE 专家负载可视化与分析}
统计推理过程中各专家被选中频率,观察是否存在“专家坍缩”现象。

\begin{figure}[H]
\centering
\includegraphics[width=0.9\linewidth]{moe_expert_load.png}
\caption{各层专家负载热力图(Top-2 选择比例)}
\end{figure}

\begin{figure}[H]
\centering
\includegraphics[width=0.8\linewidth]{moe_expert_load_overall.png}
\caption{专家整体负载分布}
\end{figure}

\textbf{分析:} 深层专家负载明显不均衡,出现坍缩趋势。建议引入 Load Balancing Loss 或路由噪声进行缓解。

\section{模型量化报告}
量化方案采用 \texttt{torch.quantization.quantize\_dynamic},仅对 \texttt{nn.Linear} 量化(W8A16)。

\begin{table}[H]
\centering
\begin{tabular}{lccccccccc}
\toprule
Model & FP32 size & INT8 size & Ratio(ckpt) & Ratio(weights) & FP32 PPL & INT8 PPL & $\Delta$PPL & FP32 latency & INT8 latency \\
\midrule
sft & 560.34 MB & 108.75 MB & 5.15x & 1.74x & 8.7782 & 8.6860 & -1.05\% & 1.698s & 1.464s \\
\bottomrule
\end{tabular}
\caption{FP32 vs INT8 大小、PPL、延迟对比表}
\end{table}

\textbf{分析:}
\begin{itemize}
\item 量化后 PPL 变化极小,精度基本保持;
\item 体积压缩明显,但对“权重-only”压缩比仅约 1.74x,原因是 embedding 等未量化;
\item CPU 推理延迟略有改善,符合动态量化预期。
\end{itemize}

\section{RAG Case Study}
使用本地知识库,采用 Ollama(qwen3:4b)进行 RAG 对比测试。

\begin{table}[H]
\centering
\begin{tabularx}{\linewidth}{lXX}
\toprule
Question & RAG OFF & RAG ON \\
\midrule
What is my student ID? & I don't know. & 123 \\
Where do I study? & I don't know. & I don't know. \\
What is my favorite topic? & I don't know. & My favorite topic is RAG. \\
\bottomrule
\end{tabularx}
\caption{RAG 开启前后回答对比}
\end{table}

\textbf{分析:} RAG 开启后能基于知识库回答事实性问题,但仍受检索召回与上下文截断影响。

\section{心得体会(润色版)}
\begin{itemize}
\item \textbf{SFT 不宜使用较长 warmup}:微调应当快速有效,长 warmup 对小数据不友好。
\item \textbf{SFT 的精髓是“微调”}:小样本 + 小学习率 + 少 epoch,避免大幅重塑模型分布。
\item \textbf{预训练 OK,微调容易跪}:学习率过大或训练过久会导致 catastrophic forgetting。
\item \textbf{语言分布一致性很关键}:预训练为纯英文,fine-tune 若混入中文会显著恶化表现。
\item \textbf{训练大模型非常不容易}:需要大量经验与反复实验去平衡收敛、泛化与遗忘。
\end{itemize}

\section{总结}
本项目完成了从架构实现、预训练、SFT、量化到 RAG 的完整流程,并提供了训练曲线、MoE 负载可视化、量化对比与 RAG Case Study 结果。虽然 SFT 表现仍有不足,但关键模块均已实现并可复现,为后续优化提供了基础。

\end{document}
